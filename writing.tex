%%%%%%%%%%%%%%%%%%%%%%%%%%%%%%%%%%%%%%%%%%%%%%%%%%%
%% LaTeX book template                           %%
%% Author:  Amber Jain (http://amberj.devio.us/) %%
%% License: ISC license                          %%
%%%%%%%%%%%%%%%%%%%%%%%%%%%%%%%%%%%%%%%%%%%%%%%%%%%

\documentclass[a4paper,11pt,UTF8]{book}
\usepackage[T1]{fontenc}
\usepackage[utf8]{inputenc}
\usepackage{lmodern}
%%%%%%%%%%%%%%%%%%%%%%%%%%%%%%%%%%%%%%%%%%%%%%%%%%%%%%%%%
% Source: http://en.wikibooks.org/wiki/LaTeX/Hyperlinks %
%%%%%%%%%%%%%%%%%%%%%%%%%%%%%%%%%%%%%%%%%%%%%%%%%%%%%%%%%
\usepackage{hyperref}
\usepackage{graphicx}
\usepackage[english]{babel}

%%%%%%%%%%%%%%%%%%%%%%%%%%%%%%%%%%%%%%%%%%%%%%%%%%%%%%%%%%%%%%%%%%%%%%%%%%%%%%%%
% 'dedication' environment: To add a dedication paragraph at the start of book %
% Source: http://www.tug.org/pipermail/texhax/2010-June/015184.html            %
%%%%%%%%%%%%%%%%%%%%%%%%%%%%%%%%%%%%%%%%%%%%%%%%%%%%%%%%%%%%%%%%%%%%%%%%%%%%%%%%
\newenvironment{dedication}
{
   \cleardoublepage
   \thispagestyle{empty}
   \vspace*{\stretch{1}}
   \hfill\begin{minipage}[t]{0.66\textwidth}
   \raggedright
}
{
   \end{minipage}
   \vspace*{\stretch{3}}
   \clearpage
}

%%%%%%%%%%%%%%%%%%%%%%%%%%%%%%%%%%%%%%%%%%%%%%%%
% Chapter quote at the start of chapter        %
% Source: http://tex.stackexchange.com/a/53380 %
%%%%%%%%%%%%%%%%%%%%%%%%%%%%%%%%%%%%%%%%%%%%%%%%
\makeatletter
\renewcommand{\@chapapp}{}% Not necessary...
\newenvironment{chapquote}[2][2em]
  {\setlength{\@tempdima}{#1}%
   \def\chapquote@author{#2}%
   \parshape 1 \@tempdima \dimexpr\textwidth-2\@tempdima\relax%
   \itshape}
  {\par\normalfont\hfill--\ \chapquote@author\hspace*{\@tempdima}\par\bigskip}
\makeatother

%%%%%%%%%%%%%%%%%%%%%%%%%%%%%%%%%%%%%%%%%%%%%%%%%%%
% First page of book which contains 'stuff' like: %
%  - Book title, subtitle                         %
%  - Book author name                             %
%%%%%%%%%%%%%%%%%%%%%%%%%%%%%%%%%%%%%%%%%%%%%%%%%%%

% Book's title and subtitle
\title{\Huge \textbf{CS Writing Phrasebook}}  %\footnote{This is a footnote.}
 %\\ \huge Sample book subtitle \footnote{This is yet another footnote.}}
% Author
\author{\textsc{Ran}\thanks{\url{https://github.com/WindChimeRan/CS-writing-phrasebook}}}

\begin{document}

\frontmatter
\maketitle

%%%%%%%%%%%%%%%%%%%%%%%%%%%%%%%%%%%%%%%%%%%%%%%%%%%%%%%%%%%%%%%
% Add a dedication paragraph to dedicate your book to someone %
%%%%%%%%%%%%%%%%%%%%%%%%%%%%%%%%%%%%%%%%%%%%%%%%%%%%%%%%%%%%%%%
% \begin{dedication}
% Dedicated to Calvin and Hobbes.
% \end{dedication}

%%%%%%%%%%%%%%%%%%%%%%%%%%%%%%%%%%%%%%%%%%%%%%%%%%%%%%%%%%%%%%%%%%%%%%%%
% Auto-generated table of contents, list of figures and list of tables %
%%%%%%%%%%%%%%%%%%%%%%%%%%%%%%%%%%%%%%%%%%%%%%%%%%%%%%%%%%%%%%%%%%%%%%%%
% \tableofcontents
% \listoffigures
% \listoftables

\mainmatter

%%%%%%%%%%%
% Preface %
%%%%%%%%%%%
% \chapter*{Preface}


%%%%%%%%%%%%%%%%
% NEW CHAPTER! %
%%%%%%%%%%%%%%%%
\chapter{Collocations}

% \begin{chapquote}{Author's name, \textit{Source of this quote}}
% ``This is a quote and I don't know who said this.''
% \end{chapquote}

\section{Verb}

% \setlength\parindent{0pt}

\begin{enumerate}

  \item We first \textbf{decided on} an inventory of semantic relations. \cite{semeval}

  \item We \textbf{accept as} relation arguments only noun
phrases with common-noun heads. \cite{semeval}

  \item This \textbf{distinguishes} our task \textbf{from} much work in Information Extraction, which tends to focus on specific classes of named entities and on considerably more fine-grained relations than we do. \cite{semeval}
  
  \item We also \textbf{impose} a syntactic locality requirement on example candidates, \textbf{thus excluding} instances where the relation arguments occur in separate sentential clauses. \cite{semeval}


\end{enumerate}

\section{Comparison}

\begin{enumerate}
  \item {It speaks to the success of} the exercise that the participating systems’ performance was \textbf{generally high, well over an order of magnitude above} random guessing. \cite{semeval}
  
  \item The best relation (presumably the easiest
  to classify) is CE, \textbf{far ahead of} ED and MC. \cite{semeval}

  \item \textbf{As compared to} traditional GNNs, GPGNNs
  could learn edges’ parameters from natural
  languages, \textbf{extending it from performing} inferring
  on only non-relational graphs or graphs with a limited
  number of edge types \textbf{to} unstructured inputs
  such as texts. \cite{GPGNN-RE}

  \item Experiment results show that our
  model \textbf{outperforms} other models \textbf{on} relation extraction
  task \textbf{by considering} multi-hop relational
  reasoning.\cite{GPGNN-RE}

  \end{enumerate}
\section{Sentence-init}

\begin{enumerate}
\item \textbf{It speaks to the success of} the exercise that the participating systems’ performance was {generally high, well over an order of magnitude above} random guessing. \cite{semeval}
\end{enumerate}

\section{Linker}

\begin{enumerate}
  \item By explicitly reasoning about missing data during learning, our approach enables large-scale training of 1D convolutional neural networks
  \textbf{while mitigating the issue of label noise inherent in distant supervision.} \cite{structured}
\end{enumerate}
%%%%%%%%%%%%%%%%%%%%%%%%%%%%%%%%%%%%%%%%%%%%%%%%%%%%%%%
% Sample table                                        %
% Source: www1.maths.leeds.ac.uk/latex/TableHelp1.pdf %
%%%%%%%%%%%%%%%%%%%%%%%%%%%%%%%%%%%%%%%%%%%%%%%%%%%%%%%
\chapter{Paragraphs}
\section{Introduction}
\begin{enumerate}

\item GP-GNNs first constructs a fully connected
      graph with the entities in the sequence
      of text. After that, it employs three modules
      to process relational reasoning: (1) an encoding
      module which enables edges to encode rich information
      from natural languages, (2) a propagation
      module which propagates relational information
      among various nodes, and (3) a classification
      module which makes predictions with node representations. \cite{GPGNN-RE}
\item 
\end{enumerate}

\section{Related Works}
\begin{enumerate}
  \item GNNs were first proposed in 2009\dots
  Later the authors in Li et al. (2016) replace\dots
  xxx propose to apply GNNs to xx, xx,\dots
  There are relatively fewer papers
  discussing how to adapt GNNs to natural
  language tasks. For example\dots
  Although they also consider applying GNNs to natural language processing
  tasks, they still perform message-passing on
  predefined graphs. Johnson (2017) introduces a
  novel neural architecture to generate a graph based
  on the textual input and dynamically update the
  relationship during the learning process. In sharp
  contrast, this paper focuses on extracting relations from real-world relation datasets. \cite{GPGNN-RE}

  \item

\end{enumerate}

\section{Methodology}
\begin{enumerate}
  \item where $f(\cdot)$ could be any model that could encode
  sequential data, such as LSTMs, GRUs, CNNs,
  $E(\cdot)$ indicates an embedding function, and $\theta$ denotes
  the parameters of the encoding module of
  n-th layer. \cite{GPGNN-RE}
\end{enumerate}
\section{Concept}

\bibliographystyle{plain}
\bibliography{bibfile}
\end{document}